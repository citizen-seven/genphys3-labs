\documentclass[a4paper, 12pt]{article}
\usepackage[T2A,T1]{fontenc}
\usepackage[utf8]{inputenc}
\usepackage[english, russian]{babel}
\usepackage{graphicx}
\usepackage[hcentering, bindingoffset = 10mm, right = 15 mm, left = 15 mm, top=20mm, bottom = 20 mm]{geometry}
\usepackage{multirow}
\usepackage{lipsum}
\usepackage{amsmath, amstext}
\usepackage{siunitx}
\usepackage{subcaption}
\usepackage{wrapfig}
\usepackage{adjustbox}
\usepackage{enumerate, indentfirst, float}
\usepackage{capt-of, svg}
\usepackage{icomma}

\newenvironment{bottompar}{\par\vspace*{\fill}}{\clearpage}
 
\begin{document}
\begin{titlepage}

\newcommand{\HRule}{\rule{\linewidth}{0.5mm}} % Defines a new command for the horizontal lines, change thickness here

\center % Center everything on the page
 
%----------------------------------------------------------------------------------------
%	HEADING SECTIONS
%----------------------------------------------------------------------------------------

\textsc{\LARGE Московский Физико-Технический Институт}\\[1,5cm] % Name of your university/college
\textsc{\Large Кафедра общей физики}\\[0.5cm] % Major heading such as course name
\textsc{\large Лабораторная работа \textnumero  3.2.1}\\[0.5cm] % Minor heading such as course title

%----------------------------------------------------------------------------------------
%	TITLE SECTION
%----------------------------------------------------------------------------------------

\HRule
\\[0.4cm]
{ \huge \bfseries Сдвиг фаз в цепи переменного тока}
\\[0.2cm] % Title of your document
\HRule
\\[1.5cm]


 
%----------------------------------------------------------------------------------------
%	AUTHOR SECTION
%----------------------------------------------------------------------------------------

\begin{minipage}{0.4\textwidth}
	\begin{flushleft} \large
		\emph{Автор:}\\
		Ришат \textsc{Исхаков} \\
		513 группа
	\end{flushleft}
\end{minipage}
~
\begin{minipage}{0.4\textwidth}
	\begin{flushright} \large
		\emph{Преподаватель:} \\
		Александр Александрович \textsc{Казимиров} % Supervisor's Name
	\end{flushright}
\end{minipage}

\begin{bottompar}
	\begin{center}
		\includegraphics[width = 80 mm]{logo.jpg}
	\end{center}
	{\large \today}

\end{bottompar}
\vfill % Fill the rest of the page with whitespace

\end{titlepage}

\section{Цель работы}
Изучить влияние активного сопротивления, индуктивности и емкости на сдвиг фаз между током и напряжением в цепи переменного тока.

\subsection*{Экспериментальная установка}



\begin {figure}[H]
\begin{center}
\includegraphics[width=0.9\textwidth]{t}
\end{center}
\end {figure}
$$R_L=50.4 \text{ Ом, при $\nu = 10 \text{ кГц}$}$$
$$L=50 \text{ мГн}$$
$$r=12.2 \text{ Ом}$$
$$C=0.5 \text{ мкФ}$$
$$\nu = 10 \text{ кГц}$$

\subsection*{RC-цепь}


Ток, текущий через RC цепочку, пропорционален напряжению на резисторе, и опережает напряжение на конденсаторе по фазе на $\pi/2$. В таком простом случае метод векторных диаграмм даёт простой результат для зависимости сдвига фаз от $R$:
$$\tg \varphi = \frac{1}{\Omega R C}$$
\begin {figure}[H]
\begin{center}
\includegraphics[width=0.6\textwidth]{rcd}
\end{center}
\end {figure}

\subsection*{RL-цепь}

Всё аналогично RC цепочке, только импеданс катушки теперь 
$$Z_2 = j\omega L,$$
поэтому ток отстаёт по фазе от напряжения, а рассчётная формула приобретает вид
$$\tg \varphi = \frac{\omega L}{R_{\sum}}$$
Теперь к споротивлению калибровочного резистора и резистора $R$ добавится активное сопротивление катушки:
$$R_{\sum} = R+r+R_L,$$
где $R_L$ -- активное сопротивление катушки.

\subsection*{RCL-цепь}

Комплексный импеданс RCL-цепочки:
$$Z=R+j\omega L - \frac{j}{\omega C}.$$

Сдвиг фаз между током и напряжением получим, взяв аргумент $Z$:

$$\tg\varphi = \frac{\omega L - \frac{1}{\omega C}}{R} = Q\frac{\left(\frac{\omega}{\omega_0}\right)^2 - 1}{\frac{\omega}{\omega_0}} = Q\frac{(1+x)^2-1}{1+x} \simeq 2x Q,$$
где $x = \Delta \omega / \omega_0 = \Delta \nu / \nu_0$, и в последнем переходе пренебрегаем квадратичными по $x$ членами.
Измерив ширину графика $w=2x$ на высоте $\varphi = \pi / 4\ (\tg\varphi = 1)$, можем непосредственно измерить добротность контура:
$$Q = \frac{1}{w}$$

\subsection*{Фазовращатель}

\begin {figure}[H]
\begin{center}
\includegraphics[width=0.8\textwidth]{phase.png}
\end{center}
\end {figure}


\begin {figure}[H]
\begin{center}
\includegraphics[width=0.7\textwidth]{diagramphase}
\end{center}
\end {figure}

Разность фаз равна $\pi /2$, когда медиана $34$ является и высотой, т.е. когда $\triangle 124$ -- равнобедренный, откуда

\section{Работа и измерения}

\subsection*{RC-цепь}

$$X_1 = \dfrac{1}{2 \pi \nu C} = 31.8$$

\begin{table}[H]
\centering
\begin{tabular}{|c|c|c|c|c|c|c|}
\hline
$R, \text{ Ом}$ & $x$ & $x_0$ & $\varphi$ & $\tg \varphi$ & $R_{\Sigma}, \text{ Ом}$ & $1/(R_{\Sigma} \Omega C)$ \\ \hline
0.0             & 1.9 & 5.0   & 1.2   & 2.5        & 12.2                  & 2.6                   \\ \hline
3.0             & 1.8 & 5.0   & 1.1   & 2.1        & 15.2                  & 2.1                   \\ \hline
5.0             & 1.7 & 5.0   & 1.1   & 1.8        & 17.2                  & 1.9                   \\ \hline
8.0             & 1.6 & 5.0   & 1.0   & 1.6        & 20.2                  & 1.6                   \\ \hline
10.0            & 1.5 & 5.0   & 0.9   & 1.4        & 22.2                  & 1.4                   \\ \hline
20.0            & 1.2 & 5.0   & 0.8   & 0.9        & 32.2                  & 1.0                   \\ \hline
30.0            & 1.0 & 5.0   & 0.6   & 0.7        & 42.2                  & 0.8                   \\ \hline
40.0            & 0.8 & 5.0   & 0.5   & 0.5        & 52.2                  & 0.6                   \\ \hline
50.0            & 0.7 & 5.0   & 0.4   & 0.5        & 62.2                  & 0.5                   \\ \hline
60.0            & 0.6 & 5.0   & 0.4   & 0.4        & 72.2                  & 0.4                   \\ \hline
80.0            & 0.5 & 5.0   & 0.3   & 0.3        & 92.2                  & 0.3                   \\ \hline
110.0           & 0.4 & 5.0   & 0.3   & 0.3        & 122.2                 & 0.3                   \\ \hline
230.0           & 0.2 & 5.0   & 0.1   & 0.1        & 242.2                 & 0.1                   \\ \hline
\end{tabular}
\caption{Полученные значения в RC-цепи}
\end{table}

Найдем погрешности измерения величин:
$$\sigma_{\tg \varphi} = 
0.1\pi \sqrt{\left(\dfrac{1}{x_0 \cos^2\left(\dfrac{\pi x}{x_0}\right)}\right)^2 + \left(\dfrac{x}{x_0^2 \cos^2 \left(\dfrac{\pi x}{x_0}\right)}\right)^2} $$

	\begin {figure}[H]
		\begin{center}
			\includegraphics[width=0.7\textwidth]{RC}
			\caption{График зависимости $\tg \varphi = f[1/ \Omega C R_{\Sigma}]$}
		\end{center}
	\end {figure}

\subsection*{RL-цепь}
$$X_2 = 2 \pi \nu L = 3231.1$$

\begin{table}[H]
\centering
\begin{tabular}{|c|c|c|c|c|c|c|}
\hline
$R, \text{ Ом}$ & $x$ & $x_0$ & $\varphi$ & $\tg \varphi$ & $R_{\Sigma}$ & $\Omega L/ R_{\Sigma} $ \\ \hline
0.0             & 2.4 & 5.0   & 1.5   & 15.9       & 62.6      & 51.6                  \\ \hline
400.0           & 2.2 & 5.0   & 1.4   & 5.2        & 462.6     & 7.0                   \\ \hline
800.0           & 2.0 & 5.0   & 1.3   & 3.1        & 862.6     & 3.7                   \\ \hline
1200.0          & 1.8 & 5.0   & 1.1   & 2.1        & 1262.6    & 2.6                   \\ \hline
1600.0          & 1.6 & 5.0   & 1.0   & 1.6        & 1662.6    & 1.9                   \\ \hline
2000.0          & 1.4 & 5.0   & 0.9   & 1.2        & 2062.6    & 1.6                   \\ \hline
2400.0          & 1.3 & 5.0   & 0.8   & 1.1        & 2462.6    & 1.3                   \\ \hline
3000.0          & 1.1 & 5.0   & 0.7   & 0.8        & 3062.6    & 1.1                   \\ \hline
\end{tabular}
\caption{Полученные значения в RL-цепи}
\end{table}

\begin {figure}[H]
	\begin{center}
		\includegraphics[width=0.65\textwidth]{RL}
		\caption{График зависимости $\tg \varphi = f[\Omega L / R_{\Sigma}]$}
	\end{center}
\end {figure}

\subsection*{RCL-цепь}

\begin{table}[H]
\centering
\begin{tabular}{|c|c|c|c|c|c|}
\hline
Сопротивление & $\nu, \text{кГц}$ & $x_0$ & $x$ & $\varphi$ & $nu\nu_0$ \\ \hline
\multirow{11}{*}{$R = 0 \; \text{Ом}$}   & 1020.0            & 2.4   & 0.2 & 0.26   & 1.01      \\ \cline{2-6} 
                                         & 1040.0            & 2.0   & 0.4 & 0.55   & 1.03      \\ \cline{2-6} 
                                         & 1060.0            & 2.0   & 0.4 & 0.63   & 1.05      \\ \cline{2-6} 
                                         & 1080.0            & 2.2   & 0.6 & 0.86   & 1.07      \\ \cline{2-6} 
                                         & 1100.0            & 2.0   & 0.6 & 0.94   & 1.09      \\ \cline{2-6} 
                                         & 1200.0            & 1.8   & 0.7 & 1.22   & 1.19      \\ \cline{2-6} 
                                         & 990.0             & 2.0   & 0.2 & 0.31   & 0.98      \\ \cline{2-6} 
                                         & 960.0             & 2.0   & 0.4 & 0.63   & 0.95      \\ \cline{2-6} 
                                         & 930.0             & 2.4   & 0.6 & 0.79   & 0.92      \\ \cline{2-6} 
                                         & 900.0             & 2.0   & 0.6 & 0.94   & 0.89      \\ \cline{2-6} 
                                         & 850.0             & 2.0   & 0.8 & 1.26   & 0.84      \\ \hline
\multirow{13}{*}{$R = 100 \; \text{Ом}$} & 1020.0            & 4.0   & 0.1 & 0.08   & 1.01      \\ \cline{2-6} 
                                         & 1040.0            & 4.0   & 0.2 & 0.16   & 1.03      \\ \cline{2-6} 
                                         & 1060.0            & 4.0   & 0.3 & 0.24   & 1.05      \\ \cline{2-6} 
                                         & 1080.0            & 4.0   & 0.4 & 0.31   & 1.07      \\ \cline{2-6} 
                                         & 1100.0            & 4.0   & 0.5 & 0.39   & 1.09      \\ \cline{2-6} 
                                         & 1160.0            & 4.0   & 0.7 & 0.55   & 1.15      \\ \cline{2-6} 
                                         & 1200.0            & 4.1   & 0.9 & 0.69   & 1.19      \\ \cline{2-6} 
                                         & 990.0             & 4.0   & 0.2 & 0.12   & 0.98      \\ \cline{2-6} 
                                         & 960.0             & 3.4   & 0.2 & 0.18   & 0.95      \\ \cline{2-6} 
                                         & 950.0             & 4.0   & 0.3 & 0.24   & 0.94      \\ \cline{2-6} 
                                         & 930.0             & 3.0   & 0.4 & 0.42   & 0.92      \\ \cline{2-6} 
                                         & 900.0             & 4.0   & 0.6 & 0.47   & 0.89      \\ \cline{2-6} 
                                         & 850.0             & 4.0   & 0.8 & 0.63   & 0.84      \\ \hline
\end{tabular}
\caption{Полученные значения при изучении зависимости фазы от $\dfrac{\nu}{\nu_0}$}
\end{table}

$C = 0.5 \text{ мкФ},$
$L =50 \text{ мГн},$
$\nu_0 = 1006.5 \text{ кГц}$

\begin {figure}[H]
	\begin{center}
		\includegraphics[width=0.8\textwidth]{R0}
		\caption{График зависимости $\varphi = f[\nu/\nu_0] \text{ для $R = 0$ Ом}$}
	\end{center}
\end {figure}

\begin {figure}[H]
	\begin{center}
		\includegraphics[width=0.8\textwidth]{R100}
		\caption{График зависимости $\varphi = f[\nu/\nu_0] \text{ для $R = 100$ Ом}$}
	\end{center}
\end {figure}

Из графика $R = 0$ Ом добротность равна:
$$Q_{0}=7 \pm1$$

Из графика $R = 100$ Ом добротность равна:
$$Q_{100}=2 \pm1$$

Можно рассчитать её, выразив через параметры цепочки:
$$Q = \frac{1}{R} \sqrt{\frac{L}{C}}$$
$$Q_{\text{теор, 0}} = 7.4$$
$$Q_{\text{теор, 100}} = 2.5$$

\section{Вывод}

На данной лабораторной работе была изучена зависимость сдвига фаз между током и напряжением от сопротивления в цепи в RC, RL, контурах. Была определена добротность колебательного контура, снята зависимость сдвига фаз от частоты вблизи резонанса. 

Для RC контура практический график довольно точно совпадает с теоретическим, однако в RL контуре значения отличаются на 20\%. Ошибка связана с неправильной установкой частоты (10 кГц вместо 1 кГц), вследствие чего изменилось и реактивное сопротивление цепи. Точнее говоря, оно стало настолько большим, что диапазон изменения $\tg \varphi$ повысился и сильно увеличилась погрешность измерения.

После изменения частоты на 1 кГц при измерении добротности колебательного контура получились достаточно точные значения, теоретические и практические совпали с учетом погрешности.

\end{document}