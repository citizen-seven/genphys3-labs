\documentclass{letnab}

\begin{document}
\begin{titlepage}
\center % Center everything on the page
 
%----------------------------------------------------------------------------------------
%	HEADING SECTIONS
%----------------------------------------------------------------------------------------

\textsc{\LARGE Московский Физико-Технический Институт}\\[1,5cm] % Name of your university/college
\textsc{\Large Кафедра общей физики}\\[0.5cm] % Major heading such as course name
\textsc{\large Лабораторная работа \textnumero  3.2.5}\\[0.5cm] % Minor heading such as course title

%----------------------------------------------------------------------------------------
%	TITLE SECTION
%----------------------------------------------------------------------------------------

\HRule
\\[0.4cm]
{ \huge \bfseries Вынужденные колебания в электрическом контуре}
\\[0.2cm] % Title of your document
\HRule
\\[1.5cm]


 
%----------------------------------------------------------------------------------------
%	AUTHOR SECTION
%----------------------------------------------------------------------------------------

\begin{minipage}{0.4\textwidth}
	\begin{flushleft} \large
		\emph{Автор:}\\
		Ришат \textsc{Исхаков} \\
		513 группа
	\end{flushleft}
\end{minipage}
~
\begin{minipage}{0.4\textwidth}
	\begin{flushright} \large
		\emph{Преподаватель:} \\
		Александр Александрович \textsc{Казимиров} % Supervisor's Name
	\end{flushright}
\end{minipage}

\begin{bottompar}
	\begin{center}
		\includegraphics[width = 80 mm]{logo.jpg}
	\end{center}
	{\large \today}

\end{bottompar}
\vfill % Fill the rest of the page with whitespace

\end{titlepage}


\section{Цель работы}
Исследование вынужденных колебаний и процессов их установления.
В работе используются: генератор звуковой частоты, осциллограф, вольтметр, частотометр, ёмкость, индуктивность, магазин сопротивлений, универсальный мост.
\section{Теоретическая часть}

В данной работе будем рассматривать колебания в электрическом колебательном контуре под воздействием внешней ЭДС, гармонически изменяющейся во времени. 
Получаем, что при подключении внешнего источника возникнут колебания, которые будем рассматривать как решение дифференциального уравнения:
\begin{equation}
L\ddot{I} + R \dot{I} + \dfrac{I}{C} = - \ef \Omega \sin \Omega t,
\end{equation}
в качестве суперпозиции двух синусоид: 
\begin{equation}
I = Be^{-\gamma t} \sin (wt-\theta) + \dfrac{\ef \Omega}{L \rho_0} \sin (\Omega t - \psi),
\end{equation}
одна из которых с частотой собственных колебаний контура $\omega$ и амплитудой, экспоненциально убывающей со временем; вторая - с частотой внешнего источника и постоянной амплитудой. Однако со временем собственные колебания затухают, и в контуре устанавливаются вынужденные колебания. А их амплитуда максимальна, когда знаменатель второй синусоиды $\rho_0 = \sqrt{(\omega_0^2 - \Omega^2_0)^2 + (2\gamma \Omega)^2}$ минимален, то есть $\omega_0 = \Omega$ (частота внешнего сигнала совпадает с собственной частотой контура). Это явление и называется \textit{резонансом}. Зависимость амплитуды колебаний от частоты внешнего напряжения называется \textit{резонансной кривой}.

\end{document}
